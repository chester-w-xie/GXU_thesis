\documentclass{article}
\usepackage{xcolor}  
\usepackage{enumitem}  
\usepackage[no-math]{fontspec}        
\usepackage{xeCJK}            
\usepackage{booktabs}   
\usepackage{amsmath}   
\usepackage{latexsym}  
\usepackage{yhmath}    
\usepackage{amssymb}   
\usepackage{eucal}   % \EuScript
\usepackage{unicode-math} 
\setmathfont{XITS Math} 
\setmathfont{XITS Math}[range={cal, bfcal}, StylisticSet=1]

% 设置正文字体
\setmainfont{Times New Roman} % 正文数字和字母使用 Times New Roman
% 设置中文正文字体为 SimSun
\setCJKmainfont{SimSun}       % 正文中文使用 SimSun
%%%%%%%%%%%%下面的定义是lshort书中符号表所需要的定义%%%%%%%%%%%%%%%%%%%%%%%%%%%%%%%
\newenvironment{symbols}[1]%
  {\small\def\arraystretch{1.5
  }
  \begin{tabular}{@{}#1@{}}}%
  {\end{tabular}}
%%%%%%%%%%%%%%%%%%%%%%%%%%%%%%%%%%%%%%%%%%%%%%%%%%%%%%%%%%%%%%%%%%%%%%%%%%%%%%%%%%%%%%%%
\begin{document}
%%%%%%%%%%%%%%%%%%%%%%%%%%%%%%%
这是正文部分,中文使用 SimSun 字体,英文使用 Times New Roman 字体。

利用unicode-math包,将表 4.2的命令改为:
\begin{table}[htp]
  \centering
  \caption{数学字母字体} \label{tbl:math-fonts}
  \begin{tabular}{*{3}{l}}
  \hline
  \textbf{示例}    & \textbf{命令} & \textbf{依赖的宏包}\\
  \hline
  $\symnormal{ABCDE abcde 1234}$  & {symnormal}\{\ldots\}&       \\
  $\symrm{ABCDE abcde 1234}$      & {symrm}\{\ldots\}    &       \\
  $\symit{ABCDE abcde 1234}$      & {symit}\{\ldots\}    &       \\
  $\symbf{ABCDE abcde 1234}$      & {symbf}\{\ldots\}    &       \\
  $\symsf{ABCDE abcde 1234}$      & {symsf}\{\ldots\}    &       \\
  $\symtt{ABCDE abcde 1234}$      & {symtt}\{\ldots\}    &       \\
  $\symcal{ABCDE}$                  & {symcal}\{\ldots\}   &   \\
  $\CMcal{ABCDE}$                  & {CMcal}\{\ldots\}   & 仅提供大写字母unicode-math下显示不同 \\
 \hline
  $\EuScript{ABCDE}$               & {mathcal}\{\ldots\}   & {eucal} 仅提供大写字母 \\
  $\symscr{ABCDE}$                & {symscr}\{\ldots\}   & \\
  $\symfrak{ABCDE abcde 1234}$    & {symfrak}\{\ldots\}  &   \\
  $\symbb{ABCDE}$                 & {symbb}\{\ldots\}    &    \\
  \hline
  \end{tabular}
\end{table}
\end{document}